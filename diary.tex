\documentclass[12pt, a4paper]{report}
\usepackage{parskip}
\usepackage[utf8]{inputenc}
\usepackage{fancyhdr}
\usepackage{pifont}
\usepackage{bibentry}
\makeatletter\let\saved@bibitem\@bibitem\makeatother
\usepackage{hyperref}
\makeatletter\let\@bibitem\saved@bibitem\makeatother
\usepackage{tikz}
\usetikzlibrary{trees,positioning,shapes,shadows,arrows}
\usepackage{amsmath}
\usepackage{amsfonts}
\usepackage{fancyvrb}
\usepackage{verbatim}
\usepackage[left=.6in,top=0.6in,right=0.5in,bottom=0.45in]{geometry}

\hypersetup{
	colorlinks=true,
	linkcolor=blue,
	filecolor=magenta,      
	urlcolor=cyan,
}

\newcommand{\stars}[1]{{\count0=#1 \loop \ifnum\count0>0 \advance\count0 by -1 \ding{80}\repeat}}
\newcommand{\record}[2]{\newpage\section*{\href{./library/#1.pdf}{#1}}\bibentry{#1}\fancyhf{}\lhead{#1}\rhead{#2}\par\bigbreak}
\newcommand{\subrecord}[1]{\\{\href{./library/#1.pdf}{#1}}\bibentry{#1}\par\bigbreak}
\pagenumbering{gobble}
\pagestyle{fancy}

\title{Reading Diary}
\author{Abu Pengwah\\The Latex-source code is proprietary to Lakshan Bernard}
\tikzset{mystyle/.style={fill=blue}}
\date{\today}
\begin{document}

\begin{titlepage}\maketitle
\end{titlepage}
\nobibliography{references.bib}
\bibliographystyle{ieeetr}
\tableofcontents{}
\newpage
\section{Unread Papers}
\VerbatimInput{filelog_unreadfiles.txt}

\chapter{Topology Identification}
\record{deka2017structure}{??/??/2020}
%-=-=-=-=-=-=-=-=-=-=-=-=-=-=-=-=-=-=-=-=-=-=-=-=-=-=-=-=-=-=-=-=-=-=-=-=-=-=-=-=-=-=-=-=-=-=-=-=-=-=-=-=
\chapter{Parameter Estimation}
\record{frigaard1995time}{14/9/2020}
It requires two synchronized measurement devices that measures the summation of the incident and reflected waves. Based on the phase shift introcuced between the signals, they manipulate the signals such that a phase shift causes the incident waves of the two measured signals to be in phase and the reflected signal to be in mutual opposite phases.

Given two measured signals, $n_1$ \& $n2$, the amplify the two signals with a constant C and phase-shift the signals by two different values $\theta_1$ and $\theta_2$. The final signal is the summation of the two new calculated signals.
The incident wave is in phase with the summated signal when some particular conditions are met. Those conditions enable the calculation of the constants C and $\theta$s.

Filters are designed to implement the phase shifting and calculation of the incident and reflected waves. Care is to be taken for singularities in the frequency.
Other knowns limitations include:
\begin{itemize}
	\item Limited Frequency Range. If the spacing between the measurement devices is too large and the frequency is also very high, the calculations of the reflections turn out to be less reliable. This is due to a decrease in the coherency factor which estimates the relative phase stability.
	\item Measurement devices spacing. If they are spaced at x/L = n/2, singularities are present in the equations.
	\item High errors due to non-linear interactions, harmonics and signal noise.
\end{itemize}
\textbf{Least Squares Method}

In this case, a 3 point method is used (i,e 3 measurement devices). The problem is formulated with the fourier coefficients of the measured signals. The number of measuring devices make the system overdetermined.

\textbf{Some issues for multi junction cables:}

All the methods suggest that it is difficult for cases where you have multiple reflections from different sources (non-linear). A method to identify which wave is the first, second, third and so on is more desirable to enhance the reliability.

\textbf{Papers using Time Domain Reflectometry to find incident and reflected waves}

The papers (proceedings letter, trans on microwave 2002) read so far use \textbf{sequential} readings to find the incident, reflected waves and the characteristic impedance.
%-=-=-=-=-=-=-=-=-=-=-=-=-=-=-=-=-=-=-=-=-=-=-=-=-=-=-=-=-=-=-=-=-=-=-=-=-=-=-=-=-=-=-=-=-=-=-=-=-=-=-=-=
\record{Cepstrum}{}
The power cepstrum of a data sequence is the square of the inverse z-transform of the logarithm of the magnitude squared of the z-transform of the data sequence. 
\begin{equation}
	C_p = [IFFT(\log |FFT(x(n))|^2)]^2
\end{equation}
Different definitions include usage of the forward z transform and/or omitting the final squaring.

For the case of a composite signal consisting of the basic wavelet and a single echo, the logarithm of the magnitude squared of the z transform of the composite signal will contain cosinusoidal ripples whose amplitude and quefrequency (frequency of the ripples) are related to the echo amplitude and delay respectively.

The presence of multiple echoes can and does confuse the interpretation procedure because of the non linearity introduced by the log function. When multiple echoes are present and the amplitude of the echo is very large, the authors propose
\begin{itemize}
	\item Power spectrum where some form of mean removal is applied before the second transformation.\item power spectrum of the square root of the power spectrum.
	\item power spectrum of the logarithm of the power spectrum.
\end{itemize}
Negative reflection coefficients give valuable information since they lead to spectral nulls. By measuring the frequency between two spectral nulls, they can determine the time delay.

Pseudo autocorrelation is defined as the inverse fourier transform of the liftered(filtered) power spectrum. If a negative reflection is present, the cepstrum will have a positive peak while the pseudo autocorrelation will have a negative peak.
This procedure can help separate multiple echoes, including multiple path reflections.

Windowing techniques could be employed to improve wavelet recovery. For low snr measurements, rectangular or Hanning window of the complex cepstrum can be used to improve the fidelity of the recovered wavelet. The authors do mention that this is probably data dependent and can also be influenced by the duration of the window. The authors mention in their summary that exponential or rectangular windows are more useful.

\textbf{Signal Detection and Extraction by Cepstrum Techniques.pdf}

They distinguish between power cepstrum and complex cepstrum. Power cepstrum is used to identify the echo arrival time. It does not have the information about the phase of the signal. The complex cepstrum gives both the information about the echo arrival time and the phase, and this is why complex cepstrum is used when the echoes or the basic wavelet are to be recovered from the composite signal.

\textit{Multiple echoes scenarios:} If the echo arrival times are different, then peaks appear in the power cepstrum at $t_0, t_1$ and $t_0 - t_1$. The amplitudes for these scenarios are hard to evaluate. They can correspond to multiple terms depending on the values of the propagation times. Hanning smoothing is once again mentioned for noisy measurements, and it improves the detection of echoes.

\textbf{Digital fourier analysis advanced techniques.pdf}

The authors look at the periodogram of a composite signal and they mention that the periodicity of the spectrum is clearer after averaging the periodogram. The logarithm based cepstrum is used for its 2 advantages:
\begin{enumerate}
	\item The decibel scale makes the small values (frequencies) more visible.
	\item Logarithm of multiplication turns out to be a summation.
\end{enumerate}
The second term in the log function is a periodic signal with the period N/d, where N is the number of samples and d is the time delay. Taking the DFT of the log function gives a peak at the reciprocal, i.e, d/N.

\textbf{Prevent Aliasing}

To avoid aliasing in the power cepstrum one must ensure that m$\tau$ $<$ T/2, where m is the order of the highest discernible term in the reflection sequence, $\tau$ is the echo delay time, and T is the record length. It is good practice to ensure that $\tau$ $<<$ T, which may be achieved either by an appropriate choice of the physical parameters of the experiment, or by zero padding the original time history. If neither of these approaches are possible, exponentially weighting the time history also reduces cepstral aliasing.
%-=-=-=-=-=-=-=-=-=-=-=-=-=-=-=-=-=-=-=-=-=-=-=-=-=-=-=-=-=-=-=-=-=-=-=-=-=-=-=-=-=-=-=-=-=-=-=-=-=-=-=-=
\record{kouassi2016reliability}{21/9/2020}
This is the first paper read on SOBOL analysis. Sobol is a global sensitivity analysis metric. There are several methods including screening methods, variance methods, nonparametric methods and density based methods. The authors focus on the variance based method. The variance of the output is given as the sum of the marginal and partial variances of the input. We have different orders of sobol sensitivity indices. The first order quantifies the main effect of the input alone on the output. The higher orders quantify the interactions between multiple terms in the input whilst not considering the single effects due to inputs alone. A large sobol index means that the corresponding random or group of components has/have a significant effect on the output Y.

Since we might have a large number of indices,, they use the term called total sensitivity indices. It is defined as the sum of all sensitivity indices related to an input. For example with 3 terms: $st_1 = s_1 + s_{12} + s_{123}$. Sobol indices are normally calculated using Monte Carlo methods.\\
%+++++++++++++++++++++++++
\subrecord{ni2017variance}
The variance based GSA is further explored. Compared to other methods, variance GSA is known to give information about the model structure. Monte Carlo simulations for large networks with many input parameters are known to be computationally intensive. The authors propose to use a different simulation technique to improve the efficiency of the system. The indices obtained from their method is easily interpreted when there is correlation between the inputs.

Under independent input uncertainties, we use the ANOVA method (variance-based). They quantify the contribution of each variable to the variance of the output. IF physical interactions among the input factors exist, the second or higher order sobol indices must be analysed. 

In case of correlated inputs, ANCOVA(covariance-based) is used. 

\textbf{A review on Global sensitivity analysis methods: Bertrand Ioss and Paul Lemaitre}

Sobol variance-based GSA is used when the model is non-linear and non-monotonic (a monotonic function is one which is either strictly increasing or decreasing; it can't be a sinusoidal like function).To get precise estimates of Sobol indices, we need a large number of model calls. $10^4$ model calls approximately gives the indices with an uncertainty of 10\%. Figures 6 \& 7 is a good descriptive figure. For cases where there is a large number of parameters, screening techniques are used to identify the inputs not contributing to the output. Once identified, these inputs are kept to their nominal values and the other inputs are given their distribution.

\textbf{Global Sensitivity Analysis: The Primer}


%-=-=-=-=-=-=-=-=-=-=-=-=-=-=-=-=-=-=-=-=-=-=-=-=-=-=-=-=-=-=-=-=-=-=-=-=-=-=-=-=-=-=-=-=-=-=-=-=-=-=-=-=
\chapter{Fault Detection}
\record{razzaghi2013efficient}{7/8/2020}
Fault event can be described as an injection in the power system in a step-like wave. Reflection coefficients, $\rho$, whose values depend on the line surge and input impedance are used. It is a step-by-step process where the fault location is guessed (and the fault impedance is apriori information), and the time-reversal approach is applied using the recorded voltages. Corresponding values of the fault currents are recorded at each guessed fault location. The fault current signal energy is calculated for all the guessed fault location, $\Gamma = \sum_{j=1}^N i^2_{x_f}(j^2)$. The maximum of the fault energies corresponds to the estimated location of the faults.
%-=-=-=-=-=-=-=-=-=-=-=-=-=-=-=-=-=-=-=-=-=-=-=-=-=-=-=-=-=-=-=-=-=-=-=-=-=-=-=-=-=-=-=-=-=-=-=-=-=-=-=-=
\record{wang2018electromagnetic}{4/9/2020}

%-=-=-=-=-=-=-=-=-=-=-=-=-=-=-=-=-=-=-=-=-=-=-=-=-=-=-=-=-=-=-=-=-=-=-=-=-=-=-=-=-=-=-=-=-=-=-=-=-=-=-=-=
\chapter{Example Chapter}

\record{einstein1935can}{27/06/2020}
I can write a summary of the paper here. I can write multiple paragraphs as follows. \par
The title of this page is the BibTeX key. The title itself is hyperlinked to the PDF of the paper. If the link does not work check the folder structure is consistent and perhaps try with a different PDF viewer (it works in Evince document viewer). When it comes time to cite this paper, I can quickly copy the BibTeX key into my LaTeX file without rummaging through the references.bib file. \par
Also note the bibliography entry is displayed above so that I can check the fields in the references.bib file have been entered properly. Since it contains indexable information (e.g. title, authors, year of publication, journal, etc.) we can use the search feature of the PDF viewer to find specific papers. \par
Each summary begins on a new page and the header also displays the BibTeX key on the left. On the right is the date that I read the paper. There is full control over the order the summaries are displayed by moving them around in the reading.tex file. I like to have a reverse chronological order since what I read recently is at the top. \par
Each paper can be assigned to a chapter for easy sorting. At the moment it is not possible to add a paper into two chapters simultaneously so I just choose the most appropriate chapter. Later on, I might make a HTML interface that allows assigning tags to each summary to make sorting more practical. \par
Since this a LaTeX document, it is quite easy to enter math mode to write equations, for example:
$$
\sin^2\theta + \cos^2\theta = 1
$$
It is also possible to have tables and even draw diagrams using TikZ.

\end{document}
